\chapter{Úvod}
\label{sec:Introduction}

V roce 2019 se celosvětově rozšířilo vysoce infekční onemocnění zvané COVID-19, které způsobuje mnoho zdravotních obtíží a má i na vině mnoho úmrtí. Vir, který je základem tohoto onemocnění, se nazývá SARS-CoV-2. Rozšíření po světě způsobilo přetrvávající celosvětovou pandemii, která zasáhla i Českou republiku. První případy se objevily 1. března 2020 a data, která poskytuje MZČR formou API, začínají od tohoto dne. Díky tomu můžeme sledovat rané začátky onemocnění COVID-19 v České republice. Poskytovaná data MZČR jsou ve formátu JSON, který je pravděpodobně neznámý běžným lidem, proto by bylo vhodné data z této textové formy nějak graficky zvizualizovat a zpřístupnit lidem k nahlédnutí.

Pro tento případ je nejvhodnější postavit webovou aplikaci, která bude jednoduchá k použití a snadno přístupná. Jediným požadavkem bude přístup k internetu, který má v dnešní době téměř každý. Škála zařízení, které mohou aplikaci spustit, je vysoká, mohou to být stolní počítače, notebooky nebo i tablety. Na mobilních telefonech se může vyskytnout problém s malou obrazovkou, kde nebudou všechny prvky webu snadno ovladatelné nebo viditelné. Dá se říci, že přístup k aplikaci má každý.

V následujících kapitolách bude probráno onemocnění COVID-19, způsoby implementace webové aplikace, aktuálně nejpopulárnější frameworky a také i srovnání s jinými weby poskytující informace o onemocnění COVID-19. Bude také detailně popsán vývoj aplikace od návrhu až po implementaci webové aplikace.

\endinput