% Nejprve uvedeme tridu dokumentu s volbami
\documentclass[czech,bachelor]{diploma}
% Dalsi doplnujici baliky maker
\usepackage[autostyle=true,czech=quotes]{csquotes} % korektni sazba uvozovek, podpora pro balik biblatex
\usepackage[backend=biber, style=iso-numeric, alldates=iso]{biblatex} % bibliografie
\usepackage{dcolumn} % sloupce tabulky s ciselnymi hodnotami
\usepackage{subfig} % makra pro "podobrazky" a "podtabulky"
\usepackage[cpp]{diplomalst}

% Zadame pozadovane vstupy pro generovani titulnich stran.
\ThesisAuthor{Radomír Čech}

\ThesisSupervisor{RNDr. Eliška Ochodková, Ph.D.}

\CzechThesisTitle{Vizualizace Covidových dat}

\EnglishThesisTitle{Visualization of Covid Data}

\SubmissionYear{2023}

\ThesisAssignmentFileName{ThesisSpecification.pdf}

% Pokud nechceme nikomu dekovat makro zapoznamkujeme.
\Acknowledgement{Rád bych na tomto místě poděkoval všem, kteří mi s prací pomohli, protože bez nich by tato práce nevznikla.}

\CzechAbstract{
    
Bakalářská práce se zabývá vývojem webové aplikace pro vizualizaci dat o onemocnění COVID-19 v České republice se zohledněním na územní členění státu (okresy). Data jsou poskytována MZČR formou webového API, které je přístupné veřejnosti.

--popis úvodu a kapitol--

}

\CzechKeywords{COVID-19; webová aplikace; API; databáze; Django framework; Django}

\EnglishAbstract{This is English abstract.}

\EnglishKeywords{typography; \LaTeX; master thesis}

\AddAcronym{API}{Application Programming Interface}
\AddAcronym{JSON}{JavaScript Object Notation}
\AddAcronym{MZČR}{Ministerstvo zdravotnictví České republiky}
\AddAcronym{COVID-19}{coronavirus disease 2019}

\addbibresource{biblatex-examples.bib}
\addbibresource{coffee.bib}

% Novy druh tabulkoveho sloupce, ve kterem jsou cisla zarovnana podle desetinne carky
\newcolumntype{d}[1]{D{,}{,}{#1}}


% Zacatek dokumentu
\begin{document}

% Nechame vysazet titulni strany.
\MakeTitlePages

% Jsou v praci obrazky? Pokud ano vysazime jejich seznam a odstrankujeme.
% Pokud ne smazeme nasledujici dve makra.
\listoffigures
\clearpage

% Jsou v praci tabulky? Pokud ano vysazime jejich seznam a odstrankujeme.
% Pokud ne smazeme nasledujici dve makra.
\listoftables
\clearpage

% A nasleduje text zaverecne prace.
\chapter{Úvod}
\label{sec:Introduction}

V roce 2019 se celosvětově rozšířilo vysoce infekční onemocnění zvané COVID-19, které způsobuje mnoho zdravotních obtíží a má i na vině mnoho úmrtí. Vir, který je základem tohoto onemocnění, se nazývá SARS-CoV-2. Rozšíření po světě způsobilo přetrvávající celosvětovou pandemii, která zasáhla i Českou republiku. První případy se objevily 1. března 2020 a data, která poskytuje MZČR formou API, začínají od tohoto dne. Díky tomu můžeme sledovat rané začátky onemocnění COVID-19 v České republice. Poskytovaná data MZČR jsou ve formátu JSON, který je pravděpodobně neznámý běžným lidem, proto by bylo vhodné data z této textové formy nějak graficky zvizualizovat a zpřístupnit lidem k nahlédnutí.

Pro tento případ je nejvhodnější postavit webovou aplikaci, která bude jednoduchá k použití a snadno přístupná. Jediným požadavkem bude přístup k internetu, který má v dnešní době téměř každý. Škála zařízení, které mohou aplikaci spustit, je vysoká, mohou to být stolní počítače, notebooky nebo i tablety. Na mobilních telefonech se může vyskytnout problém s malou obrazovkou, kde nebudou všechny prvky webu snadno ovladatelné nebo viditelné. Dá se říci, že přístup k aplikaci má každý.

V následujících kapitolách bude probráno onemocnění COVID-19, způsoby implementace webové aplikace, aktuálně nejpopulárnější frameworky a také i srovnání s jinými weby poskytující informace o onemocnění COVID-19. Bude také detailně popsán vývoj aplikace od návrhu až po implementaci webové aplikace.

\endinput
\input{Chapters/SampleChapter1.tex}
\input{Chapters/SampleChapter2.tex}
\input{Chapters/TechnicalDetails.tex}
\input{Chapters/Conclusion.tex}

% Seznam literatury
\printbibliography[title={Literatura}, heading=bibintoc]

% Prilohy
\appendix
\input{Chapters/Appendix1.tex}
\input{Chapters/Appendix2.tex}

% Priloha vlozena primo do hlavniho LaTeX souboru. Ne vsechny prilohy je nutne mit ve zvlastnich souborech.
\chapter{Dlouhý zdrojový kód}
\lstinputlisting[label=src:CppExternal,caption={Dlouhý zdrojový kód v jazyce C++ načtený s externího souboru}]{SourceCodes/ArraySortingAlgorithms.cpp}

\end{document}
